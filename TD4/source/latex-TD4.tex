documentclass{article}
usepackage{latexsym,amssymb,enumerate,amsmath,epsfig,amsthm}
usepackage[margin=1in]{geometry}
usepackage{setspace,color}
usepackage{graphicx}
usepackage{subfigure}
usepackage[english]{babel}
usepackage[table,xcdraw]{xcolor}
usepackage[utf8]{inputenc}
usepackage{amsmath}
usepackage{graphicx}
usepackage[colorinlistoftodos]{todonotes}
usepackage{geometry}
usepackage{caption}
usepackage{url}
usepackage{array}
usepackage[toc,page]{appendix}

usepackage{xcolor}
usepackage{listings}

usepackage{tikz}
usetikzlibrary{shapes,arrows}
tikzstyle{block} = [rectangle, draw, text width=7.5em, text centered, rounded corners,node distance=4cm, minimum height=4em]
tikzstyle{line} = [draw, -latex']

newtheorem{eg}{Example}[section]
newcommand{ds}{displaystyle}
usepackage{hyperref}
usepackage{xcolor}
hypersetup{
    colorlinks,
    linkcolor={red!50!black},
    citecolor={blue!50!black},
    urlcolor={blue!80!black}
}


definecolor{mGreen}{rgb}{0,0.6,0}
definecolor{mGray}{rgb}{0.5,0.5,0.5}
definecolor{mPurple}{rgb}{0.58,0,0.82}
definecolor{backgroundColour}{rgb}{0.95,0.95,0.92}
lstdefinestyle{CStyle}{
    backgroundcolor=color{backgroundColour},   
    commentstyle=color{mGreen},
    keywordstyle=color{magenta},
    numberstyle=tinycolor{mGray},
    stringstyle=color{mPurple},
    basicstyle=footnotesize,
    breakatwhitespace=false,         
    breaklines=true,                 
    captionpos=b,                    
    keepspaces=true,                 
    numbers=left,                    
    numbersep=5pt,                  
    showspaces=false,                
    showstringspaces=false,
    showtabs=false,                  
    tabsize=2,
    language=C
}

begin{document}
title{Complexité et Calculabilité}
author{
Groupe D , L3-Informatique }

maketitle



section{Partition en Triangles }
subsection{Introduction }
subsubsection{Langage }
Toute traduction elle s'opère entre un langage formel $sigma $ et un langage $theta$ .
Un langage est un ensemble de mot bien formé , pour savoir si un mot est bien formé nous devons doté notre langage de règle de grammaire .

Un mot est un ensemble de symbole , les symboles sont définit sur un alphabet .

subsubsection{Problème de décision}

Un type de problème est un ensemble de phrase obéissant à une certaine construction , par exemple  toute les phrases qui obéissent à la forme Fait il beau dans le pays x  , avec x $in$ Pays  

Un problème de décision est un problème qui demande de répondre à la phrase par oui ou non .

Ainsi un problème de décision peut être vue comme un langage L tel que l'ensemble des phrases appartenant au langage L est associé à une réponse positive . 


subsubsection{Traduction }


Une réduction est appelé communément appelé une traduction .   

Ainsi l'on  dit qu'on a réduit un problème A à un problème B si l'on arrive à trouver une fonction qui traduit tout instance d'un problème A en une instance du problème B . La réduction se note  $A leq B$ .

Comme dit précédemment un problème est un langage d'où le mot traduction, ainsi f se définit par 

Soit A un problème  définit sur un langage $sigma$ et B un problème NP-complet  définit sur un langage $theta$ .

Si f est une fonction de traduction de A vers B alors on a 

$f  sigma^ rightarrow theta^$ tel que 
$A leq B$ $leftrightarrow$ 
$w in A leftrightarrow f(w) in B  $

En francais ça donne 

Soit f tel que  w décrit une instance du problème A  $leftrightarrow$  on peut traduire w en une instance du problème B par f  .


Une autre façon de voir les choses est de dire que underline{tout problème de type A} est mis en correspondance par f avec une partie des problèmes de type B .

subsubsection{Réduction }

Remarque  f est supposé est calculable en temps polynomial . 

Si nous avons un algorithme  Solvetextunderscore B nous répondant oui ou non pour chaque instance d'un problème de type B et que nous avons f underline{une fonction totale},  traduisant un problème de A vers un problème de type B .
Alors comme f est totale , $forall a in L(A)$ Solvetextunderscore B ( $f(I_a)$ ) résout tous les problème de A . 


Ainsi on pourrait dire que f(A) est inclu dans L(B) et comme f est totale L(A) est inclu dans L(B) à une traduction près . Il serait ainsi naturel de dire que B est au moins aussi dur que A car B contient la complexité de A (à un temps polynomial près ) .


section{Exerice}


Notre objéctif va être de prouver que le problème de la partition en Triangles est NP-complet .


subsection{Probleme Enonce}
 title{Partition en Triangles}

textbf{Entrée}  Un graphe G = (V,E) avec $V = 3q, q in  mathbb{N}$.

medskip

underline{Question}  Est-ce qu’il existe une partition de V en q ensembles disjoints V1, V2,..., Vq de trois sommets chacun tel que pour chaque $Vi = {v_{i1}, v_{i2} , v_{i3}}$ les trois arêtes appartiennent à E 
medskip

Autrement dit peut-on couvrir le graphe par des triangles de taille trois 


subsection{Préliminaire}
Pour prouver que le problème P1 est NP-complet on doit réduire  un problème connu comme étant NP-Complet à P1 , ainsi on prouvera que P1 est au moins aussi dur qu'un problème NP et par transitivité qu'il est NP-complet

 En effet , tout problème de NP est réductible à un problème NP-complet grâce à une certaine fonction de traduction $f_i$ 

Soit PC1 un problème NP-Complet et $L(PC1)$ son langage , alors nous avons 
$forall P_i in NP exists f_i in poly_function forall I_i in L(P_i)$ tq $f_i(I_i) in L(PC1)$ 

Si un problème NP-complet PC1 se réduit à un problème NP1 par une fonction de traduction g1 

$exists g1 in poly_function forall IC_i in L(PC1)$ tq $g1(IC_i) in L(NP1)$ 

Alors on a 

$forall I_i in L(P_i)$ $g1(f_i(I_i)) in L(NP1)$ 

Donc NP1 est aussi NP-complet .

section{Résolution}
Nous allons utilisé le problème 3-SAT pour la réduction.Ainsi nous devons traduire une instance de 3-SAT en une instance de la partition Triangle.

subsection{Traduction }

subsubsection{Vocabulaire}

Soit $mathbb{P}$ un ensemble de  variables porpositionnelle .

Un littéral est de la forme  $forall l_i, exists pv_i in mathbb{V} $ where $l_i in {lnot pv_i ,pv_i}$ 


Une clause est de la forme  $forall C_i , C_i= bigvee_{i in [r]}l_i $ où  $l_i in mathbb{L} $ et $r in mathbb{N}$

Et soit $taille  mathbb{C} rightarrow mathbb{N}$ qui associe à chaque clause le nombre de littéraux dans celle ci . 

Notation  Soit $C_i$ une clause quelconque $taille(C_i)=C_i$

medskip

Une CNF est une conjonction de clause 

$bigwedge_{i in [r]}C_i$ avec $C_i in mathbb{C}$ et $C_i = r$

Une 3-CNF est définit comme suit
$bigwedge_{i in [3] }C_i$ avec $C_i in mathbb{C}$ et $C_i=3$

subsection{3-SAT}


Soit $phi$ une 3-CNF définit sur $mathbb{V}$  ( on suppose que l'inteprétation des symboles de $phi$ est hérité de la logique classique , on notera J cette structure ).


Une instance du problème 3-SAT est définit comme suit 

Existe-t-il une assignation qui satisfasse $phi$ 

$phi^{mathbb{J}}[l_0rightarrow b_0,l_1rightarrow b_1,...,l_nrightarrow b_n] = 1$ avec $l_i in mathbb{L} $ et $b_i in {0,1} $

Ou de manière fonctionnelle avec $theta  mathbb{P}^n rightarrow {{0,1}}^n$
tel que $forall l_i in {l_0,l_1,...,l_p} ,theta_{b_0,b_1,...,b_p}(l_i)=b_i$ et $theta_{b_0,b_1,...,b_p}^{phi}(phi) = 1 $

(Cette fonction pourrait nous être utile pour résoudre FSAT ).

subsection{Traduction  }
Soit l'instance d'un prolbème 3-SAT avec n clauses et k variables .
On note $phi$ l'expression à satisfaire , $mathbb{V}^{phi}$ l'ensemble des symboles de variables de $phi$, $mathbb{L}^{phi}$ l'ensemble des littéraux associer à $mathbb{V}^{phi}$ ,$mathbb{C}^{phi}$ l'ensemble des clauses appareceant dans $phi$ .


(On utilisera 
$lp_i et ln_i in mathbb{L}^{phi}$ , $v_i in mathbb{V}^{phi}$  et $C_i in mathbb{C}^{phi}$  ) 


Les littéraux de $mathbb{L}^{phi}$  seront traduits par les sommets ${s_1,...,s_2kn}$ . 
En sommes on crée pour chaque variable de $mathbb{V}^{phi}$ (k) 2 sommets (2) pour chacune des n clauses de $phi$(n) (=2kn).

Le premier sommet $s_{2i}$ représente la pontentielle occurence du littéral positif $lp_i=v_i$ dans la clause $C_i$ et l'autre  $s_{2i+1}$ la pontentielle occurence du littéral négatif  $ln_i=lnot v_i$ dans la clause $C_i$ .


Soit un ensemble de sommets  ${a_1,...,a_{kn}}$ , taille = kn car il y a kn sommets invalidé par l'assignation parmis les 2kn . 

La distribution de valeurs de vérité $theta$ sera traduite par 
si $thetal^{phi}(lp_i) =1$ avec $lp_i in mathbb{L} $ un littéral positif  alors on ajoutera le triplet $(s_{2i+1},a_{2i},a_{2i+1})$ à la couverture $Theta$ .
sinon on ajoutera le triplet $(s_{2i},a_{2i},a_{2i+1})$ à la couverture $Theta$ .
Ces triplets représente  les littérauxclauses qui ont été exclu par la nature de l'assignation .

Soit un ensemble de variable  ${a_0,a_1,...,a_{2n}}$ 

Soit un ensemble de variable  ${c_1,...,c_{3n}} in {0,1}^{3n}$ , $c_{3i}=1$ si la première variable de la clause $C_i$  est un littéral positif sinon $c_{3i}=0$.


La clause $C_i$ sera traduite par 3 triplets , dont chacun possède 2 sommets dans $ {a_1,...,a_{2n}} $ et un sommet dans ${s_1,...,s_{2kn}} $ .
Ainsi $C_i$ se transforme en 

$CT_i={(s_{2ntimes n_1+i+c_{3i}},a_{2i},a_{2i+1}),(s_{2ntimes n_2+i+c_{1+3i}},a_{2i},a_{2i+1}),(s_{2ntimes n_3+i+c_{2+3i}},a_{2i},a_{2i+1})}$

avec $n_1$ ,$n_2$ et $n_3$ les indexes des variables apparaissant dans la clause$ C_i$ . ( voir exemple).

Si $exists e in [2kn] $ tel que $s_{e}$ n'est pas encore couvert alors $forall CT_i $ tel que $s_{e} in CT_i $  on ajoute les triplets de $ CT_i$ à $Theta$ .


Soit $C_i=(l_1^i ,l_2^i,l_3^i),theta^{C_i}(C_i)=1 $  sera traduit par $exists e in [2kn] $ tel que  $(s_{e},a_{2i},a_{2i+1}) in Theta $

Soit $theta^{phi}(phi)=1 $  sera traduit par $forall C_i in mathbb{C} , exists e in [2kn] $ tel que  $(s_{e},a_{2i},a_{2i+1}) in Theta $ ou plus simplement $Theta$ couvre tous les points du graphe . 


Ainsi par exemple $2ntimes n_1$ nous place sur le premier sommet de la variable numéro $n_1$ puis $+i$ sur l'instance positive destiné à la clause $C_i$ et $+c_{3i}$ sur l'instance appareceant dans $C_i$ ( +0 si positif +1 si négatif).


end{document}